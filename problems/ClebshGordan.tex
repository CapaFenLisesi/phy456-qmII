%
% Copyright � 2012 Peeter Joot.  All Rights Reserved.
% Licenced as described in the file LICENSE under the root directory of this GIT repository.
%

\label{chap:ClebshGordan}
%\blogpage{http://sites.google.com/site/peeterjoot2/math2011/ClebshGordan.pdf}
%\date{Nov 23, 2011}

\section{Motivation}

In \S 28.2 of the text \citep{desai2009quantum} is a statement that the Clebsh-Gordan coefficient
\index{Clebsh-Gordan coefficient}

\begin{equation}\label{eqn:ClebshGordan:10}
\braket{m_1 m_2}{jm}
\end{equation}

unless \(m = m_1 + m_2\).  It appeared that it was related to the operation of \(J_z\), but how exactly was not obvious to me.  In tutorial today we hashed through this.  Here is the details lying behind this statement

\section{Recap on notation}

We are taking an arbitrary two particle ket and decomposing it utilizing an insertion of a complete set of states

\begin{equation}\label{eqn:ClebshGordan:30}
\ket{jm} = \sum_{m_1' m_2'}
\Bigl(
\ket{j_1 m_1'} \ket{j_2 m_2'}
\bra{j_1 m_1'} \bra{j_2 m_2'}
\Bigr)
\ket{jm}
\end{equation}

with \(j_1\) and \(j_2\) fixed, this is written with the shorthand

\begin{equation}\label{eqn:ClebshGordan:50}
\begin{aligned}
\ket{j_1 m_1} \ket{j_2 m_2} &= \ket{m_1 m_2} \\
\bra{j_1 m_1} \bra{j_2 m_2} \ket{jm} &= \braket{m_1 m_2}{jm},
\end{aligned}
\end{equation}

so that we write

\begin{equation}\label{eqn:ClebshGordan:70}
\ket{jm} = \sum_{m_1' m_2'} \ket{m_1' m_2'} \braket{m_1' m_2'}{jm}
\end{equation}

\section{The \texorpdfstring{\(J_z\)}{J z} action}

We have two ways that we can apply the operator \(J_z\) to \(\ket{jm}\).  One is using the sum above, for which we find

\begin{equation}\label{eqn:ClebshGordan:130}
\begin{aligned}
J_z \ket{jm}
&= \sum_{m_1' m_2'} J_z \ket{m_1' m_2'} \braket{m_1' m_2'}{jm} \\
&= \Hbar \sum_{m_1' m_2'} (m_1' + m_2') \ket{m_1' m_2'} \braket{m_1' m_2'}{jm} \\
\end{aligned}
\end{equation}

We can also act directly on \(\ket{jm}\) and then insert a complete set of states

\begin{equation}\label{eqn:ClebshGordan:150}
\begin{aligned}
J_z \ket{jm}
&=
\sum_{m_1' m_2'}
\ket{m_1' m_2'}
\bra{m_1' m_2'}
J_z \ket{jm} \\
&=
\Hbar m
\sum_{m_1' m_2'}
\ket{m_1' m_2'}
\braket{m_1' m_2'}{jm}
\\
\end{aligned}
\end{equation}

This provides us with the identity

\begin{equation}\label{eqn:ClebshGordan:90}
m
\sum_{m_1' m_2'}
\ket{m_1' m_2'}
\braket{m_1' m_2'}{jm}
= \sum_{m_1' m_2'} (m_1' + m_2') \ket{m_1' m_2'} \braket{m_1' m_2'}{jm}
\end{equation}

This equality must be valid for any \(\ket{jm}\), and since all the kets \(\ket{m_1' m_2'}\) are linearly independent, we must have for any \(m_1', m_2'\)

\begin{equation}\label{eqn:ClebshGordan:110}
(m - m_1' - m_2') \braket{m_1' m_2'}{jm} \ket{m_1' m_2'} = 0
\end{equation}

We have two ways to get this zero.  One of them is a \(m = m_1' + m_2'\) condition, and the other is for the CG coeff \(\braket{m_1' m_2'}{jm}\) to be zero whenever \(m \ne m_1' + m_2'\).

It is not a difficult argument, but one that was not clear from a read of the text (at least to me).


