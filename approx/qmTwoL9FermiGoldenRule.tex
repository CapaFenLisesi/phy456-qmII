%
% Copyright � 2013 Peeter Joot.  All Rights Reserved.
% Licenced as described in the file LICENSE under the root directory of this GIT repository.
%

%\section{Fermi's golden rule}
\index{Fermi's golden rule}

See \S 17.2 of the text \citep{desai2009quantum}.

Fermi originally had two golden rules, but his first one has mostly been forgotten.  This refers to his second.

This is really important, and probably the \textunderline{single most} important thing to learn in this course.  You will find this falls out of many complex calculations.

Returning to general time dependent equations with

\begin{equation}\label{eqn:qmTwoL9FermiGoldenRule:330}
H = H_0 + H'(t)
\end{equation}

\begin{equation}\label{eqn:qmTwoL9FermiGoldenRule:350}
\ket{\psi(t)} = \sum_n c_n(t) e^{-i\omega_n t} \ket{\psi_n}
\end{equation}

and

\begin{equation}\label{eqn:qmTwoL9FermiGoldenRule:370}
i \Hbar \dot{c}_n = \sum_n H_{mn}' e^{i \omega_{mn} t} c_n(t)
\end{equation}

where

\begin{equation}\label{eqn:qmTwoL9FermiGoldenRule:390}
\begin{aligned}
H_{mn}'(t) &= \bra{\psi_m} H'(t) \ket{\psi_n} \\
\omega_n &= \frac{E_n}{\Hbar} \\
\omega_{mn} &= \omega_m - \omega_n
\end{aligned}
\end{equation}

\makeexample{Electric field potential}{l9:ex1}{

\begin{equation}\label{eqn:qmTwoL9FermiGoldenRule:410}
H'(t) = - \Bmu \cdot \BE(t).
\end{equation}

If \(c_m^{(0)} = \delta_{mi}\), then to first order

\begin{equation}\label{eqn:qmTwoL9FermiGoldenRule:430}
i \Hbar \dot{c}^{(1)}(t) = H_{mi}'(t) e^{i \omega_{mi} t},
\end{equation}

and

\begin{equation}\label{eqn:qmTwoL9FermiGoldenRule:450}
c_m^{(1)}(t) = \inv{i\Hbar} \int_{t_0}^t H_{mi}'(t') e^{i \omega_{mi} t'} dt'.
\end{equation}

Assume the perturbation vanishes before time \(t_0\).

\paragraph{Reminder}.  Have considered this using \eqnref{eqn:qmTwoL9FermiGoldenRule:450} for a pulse as in \cref{fig:qmTwoL9FermiGoldenRule:gaussianWavePacket}

\imageFigure{../figures/phy456-qmII/gaussianWavePacket}{Gaussian wave packet}{fig:qmTwoL9FermiGoldenRule:gaussianWavePacket}{0.2}

Now we want to consider instead a non-terminating signal, that was zero before some initial time as illustrated in \cref{fig:qmTwoL9FermiGoldenRule:unitStepSine}, where the separation between two peaks is \(\Delta t = 2\pi/\omega_0\).

\imageFigure{../figures/phy456-qmII/unitStepSine}{Sine only after an initial time}{fig:qmTwoL9FermiGoldenRule:unitStepSine}{0.2}

Our matrix element is

\begin{equation}\label{eqn:qmTwoL9FermiGoldenRule:470}
H_{mi}'(t) = - \bra{\psi_m} \Bmu \ket{\psi_i} \cdot \BE(t) =
\left\{
\begin{array}{l l}
2 A_{mi} \sin(\omega_0 t) & \quad \mbox{if \(t > 0\)} \\
0 & \quad \mbox{if \(t < 0\)} \\
\end{array}
\right.
\end{equation}

Here the factor of 2 has been included for consistency with the text.

\begin{equation}\label{eqn:qmTwoL9FermiGoldenRule:490}
H_{mi}'(t) = i A_{mi}
\left(
e^{-i \omega_0 t}
-e^{i \omega_0 t}
\right)
\end{equation}

Plug this into the perturbation

\begin{equation}\label{eqn:qmTwoL9FermiGoldenRule:510}
c_m^{(1)}(t) =
\frac{A_{mi}}{\Hbar} \int_{t_0}^t dt'
\left(
e^{i (\omega_{mi} - \omega_0 t) }
-e^{i (\omega_{mi} + \omega_0 t) }
\right)
\end{equation}

\pdfTexFigure{../figures/phy456-qmII/qmTwoL9fig6.pdf_tex}{\(\omega_{mi}\) illustrated}{fig:qmTwoL9FermiGoldenRule:6}{0.3}
%\cref{fig:qmTwoL9FermiGoldenRule:6}

Suppose that

\begin{equation}\label{eqn:qmTwoL9FermiGoldenRule:530}
\omega_0 \approx \omega_{mi},
\end{equation}

then

\begin{equation}\label{eqn:qmTwoL9FermiGoldenRule:550}
c_m^{(1)}(t) \approx
\frac{A_{mi}}{\Hbar} \int_{t_0}^t dt'
\left(
1
-e^{2 i \omega_0 t }
\right),
\end{equation}

but the exponential has essentially no contribution

\begin{equation}\label{eqn:qmTwoL9FermiGoldenRule:630}
\begin{aligned}
\Abs{\int_0^t e^{2 i \omega_0 t'} dt' }
&=
\Abs{\frac{e^{2 i \omega_0 t} -1 }{2 i \omega_0}}  \\
&=
\frac{\sin(\omega_0 t)}{\omega_0} \\
&\sim \inv{\omega_0}
\end{aligned}
\end{equation}

so for \(t \gg \inv{\omega_0}\) and \(\omega_0 \approx \omega_{mi}\) we have

\begin{equation}\label{eqn:qmTwoL9FermiGoldenRule:570}
c_m^{(1)}(t) \approx \frac{A_{mi}}{\Hbar} t
\end{equation}

Similarly for \(\omega_0 \approx \omega_{im}\) as in \cref{fig:qmTwoL9FermiGoldenRule:7}

\pdfTexFigure{../figures/phy456-qmII/qmTwoL9fig7.pdf_tex}{FIXME: qmTwoL9fig7}{fig:qmTwoL9FermiGoldenRule:7}{0.3}

then

\begin{equation}\label{eqn:qmTwoL9FermiGoldenRule:590}
c_m^{(1)}(t) \approx
\frac{A_{mi}}{\Hbar} \int_{t_0}^t dt'
\left(
e^{-2 i \omega_0 t }
-1
\right),
\end{equation}

and we have

\begin{equation}\label{eqn:qmTwoL9FermiGoldenRule:610}
c_m^{(1)}(t) \approx -\frac{A_{mi}}{\Hbar} t
\end{equation}
}

\shipoutAnswer
