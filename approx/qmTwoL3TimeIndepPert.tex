%
% Copyright � 2013 Peeter Joot.  All Rights Reserved.
% Licenced as described in the file LICENSE under the root directory of this GIT repository.
%
See \S 16.1 of the text \citep{desai2009quantum}.

We can sometimes use this sort of physical insight to help construct a good approximation.  This is provided that we have some of this physical insight, or that it is good insight in the first place.

This is the no-think (turn the crank) approach.

Here we split our Hamiltonian into two parts

\begin{equation}\label{eqn:qmTwoL3TimeIndepPert:310}
H = H_0 + H'
\end{equation}

where \(H_0\) is a Hamiltonian for which we know the energy eigenstates and the eigenkets.  The \(H'\) is the ``perturbation'' that is supposed to be small ``in some sense''.

Prof Sipe will provide some references later that provide a more specific meaning to this ``smallness''.  From some ad-hoc discussion in the class it sounds like one has to consider sequences of operators, and look at the convergence of those sequences (is this L2 measure theory?)

\imageFigure{../../figures/phy456/qmTwoL3fig3}{Example of small perturbation from known Hamiltonian}{fig:qmTwoL3TimeIndepPert:3}{0.4}
%\cref{fig:qmTwoL3TimeIndepPert:3}

We would like to consider a range of problems of the form

\begin{equation}\label{eqn:qmTwoL3TimeIndepPert:330}
H = H_0 + \lambda H'
\end{equation}

where

\begin{equation}\label{eqn:qmTwoL3TimeIndepPert:350}
\lambda \in [0,1]
\end{equation}

So that when \(\lambda \rightarrow 0\) we have

\begin{equation}\label{eqn:qmTwoL3TimeIndepPert:370}
H \rightarrow H_0
\end{equation}

the problem that we already know, but for \(\lambda \rightarrow 1\) we have

\begin{equation}\label{eqn:qmTwoL3TimeIndepPert:390}
H = H_0 + H'
\end{equation}

the problem that we would like to solve.

We are assuming that we know the eigenstates and eigenvalues for \(H_0\).  Assuming no degeneracy

\begin{equation}\label{eqn:qmTwoL3TimeIndepPert:410}
H_0 \ket{\psi_s^{(0)}} =
E_s^{(0)}
\ket{\psi_s^{(0)}}
\end{equation}

We seek
\begin{equation}\label{eqn:qmTwoL3TimeIndepPert:430}
(H_0 + H')\ket{\psi_s} =
E_s
\ket{\psi_s}
\end{equation}

(this is the \(\lambda = 1\) case).

Once (if) found, when \(\lambda \rightarrow 0\) we will have

\begin{equation}\label{eqn:qmTwoL3TimeIndepPert:550}
\begin{aligned}
E_s &\rightarrow E_s^{(0)} \\
\ket{\psi_s} &\rightarrow \ket{\psi_s^{(0)}}
\end{aligned}
\end{equation}

\begin{equation}\label{eqn:qmTwoL3TimeIndepPert:450}
E_s = E_s^{(0)}  + \lambda E_s^{(1)} + \lambda^2 E_s^{(2)}
\end{equation}

\begin{equation}\label{eqn:qmTwoL3TimeIndepPert:470}
\psi_s = \sum_n c_{ns} \ket{\psi_n^{(0)}}
\end{equation}

This we know we can do because we are assumed to have a complete set of states.

with
\begin{equation}\label{eqn:qmTwoL3TimeIndepPert:490}
c_{ns} = c_{ns}^{(0)}  + \lambda c_{ns}^{(1)} + \lambda^2 c_{ns}^{(2)}
\end{equation}

where

\begin{equation}\label{eqn:qmTwoL3TimeIndepPert:510}
c_{ns}^{(0)} = \delta_{ns}
\end{equation}

There is a subtlety here that will be treated differently from the text.  We write

\begin{equation}\label{eqn:qmTwoL3TimeIndepPert:570}
\begin{aligned}
\ket{\psi_s}
&=
\ket{\psi_s^{(0)}}
+
\lambda
\sum_n
c_{ns}^{(1)}
\ket{\psi_n^{(0)}}
+
\lambda^2
\sum_n
c_{ns}^{(2)}
\ket{\psi_n^{(0)}}
+ \cdots \\
&=
\left(
1 + \lambda c_{ss}^{(1)} + \cdots
\right)
\ket{\psi_s^{(0)}}
+ \lambda
\sum_{n \ne s} c_{ns}^{(1)}
\ket{\psi_n^{(0)}}
+ \cdots
\end{aligned}
\end{equation}

Take
\begin{equation}\label{eqn:qmTwoL3TimeIndepPert:590}
\begin{aligned}
\ket{\overbar{\psi}_s}
&=
\ket{\overbar{\psi}_s^{(0)}}
+
\lambda
\frac{
\sum_{n \ne s} c_{ns}^{(1)}
\ket{\psi_n^{(0)}}
}
{
1 + \lambda c_{ss}^{(1)}
}
+ \cdots
\\
&=
\ket{\overbar{\psi}_s^{(0)}}
+
\lambda
\sum_{n \ne s} \overbar{c}_{ns}^{(1)}
\ket{\psi_n^{(0)}} + \cdots
\end{aligned}
\end{equation}

where

\begin{equation}\label{eqn:qmTwoL3TimeIndepPert:610}
\overbar{c}_{ns}^{(1)}  =
\frac{c_{ns}^{(1)} }
{
1 + \lambda c_{ss}^{(1)}
}
\end{equation}

We have:

\begin{equation}\label{eqn:qmTwoL3TimeIndepPert:630}
\begin{aligned}
\overbar{c}_{ns}^{(1)} &= c_{ns}^{(1)} \\
\overbar{c}_{ns}^{(2)} &\ne c_{ns}^{(2)}
\end{aligned}
\end{equation}

FIXME: I missed something here.

Note that this is no longer normalized.

\begin{equation}\label{eqn:qmTwoL3TimeIndepPert:530}
\braket{\overbar{\psi}_s}{\overbar{\psi}_s} \ne 1
\end{equation}


