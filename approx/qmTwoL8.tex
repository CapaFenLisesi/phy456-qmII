%
% Copyright � 2012 Peeter Joot.  All Rights Reserved.
% Licenced as described in the file LICENSE under the root directory of this GIT repository.
%

%
%
%\input{../peeter_prologue_print.tex}
%\input{../peeter_prologue_widescreen.tex}

%\usepackage{pstricks}
%\usepackage[off]{auto-pst-pdf}
% for:
%\input{qmTwoL8fig0FrequenciesAbsorbtionAndEmission}
% as generated by Inkscape.
%
% not sure how to make this work?

%\chapter{PHY456H1F: Quantum Mechanics II.  Lecture 8 (Taught by Prof J.E. Sipe).  Time dependent perturbation (cont.)}
\index{time dependent perturbation}
%\chapter{Time dependent perturbation (cont.)}
\label{chap:qmTwoL8}
\blogpage{http://sites.google.com/site/peeterjoot/math2011/qmTwoL8.pdf}
%\date{Oct 3, 2011}





\section{Time dependent perturbation}

We would gotten as far as calculating

\begin{equation}\label{eqn:qmTwoL8:10}
c_m^{(1)}(\infty) = \inv{i \Hbar} \Bmu_{ms} \cdot \BE(\omega_{ms})
\end{equation}

where

\begin{equation}\label{eqn:qmTwoL8:30}
\BE(t) = \int \frac{d\omega}{2 \pi} \BE(\omega) e^{-i \omega t},
\end{equation}

and

\begin{equation}\label{eqn:qmTwoL8:50}
\omega_{ms} = \frac{E_m - E_s}{\Hbar}.
\end{equation}

Graphically, these frequencies are illustrated in \cref{fig:qmTwoL8fig0FrequenciesAbsorbtionAndEmission}

\pdfTexFigure{../../figures/phy456/qmTwoL8fig0FrequenciesAbsorbtionAndEmission.pdf_tex}{Positive and negative frequencies}{fig:qmTwoL8fig0FrequenciesAbsorbtionAndEmission}{0.3}

The probability for a transition from \(m\) to \(s\) is therefore

\begin{equation}\label{eqn:qmTwoL8:70}
\rho_{m \rightarrow s} = \Abs{ c_m^{(1)}(\infty) }^2
= \inv{\Hbar}^2 \Abs{\Bmu_{ms} \cdot \BE(\omega_{ms})}^2
\end{equation}

Recall that because the electric field is real we had

\begin{equation}\label{eqn:qmTwoL8:90}
\Abs{\BE(\omega)}^2 = \Abs{\BE(-\omega)}^2.
\end{equation}

Suppose that we have a wave pulse, where our field magnitude is perhaps of the form

\begin{equation}\label{eqn:qmTwoL8:110}
E(t) = e^{-t^2/T^2} \cos(\omega_0 t),
\end{equation}

as illustrated with \(\omega = 10, T = 1\) in \cref{fig:qmTwoL8:gaussianWavePacket}.

\imageFigure{../../figures/phy456/gaussianWavePacket}{Gaussian wave packet}{fig:qmTwoL8:gaussianWavePacket}{0.2}

We expect this to have a two lobe Fourier spectrum, with the lobes centered at \(\omega = \pm 10\), and width proportional to \(1/T\).

For reference, as calculated using \nbref{qmTwoL8figures.nb} this Fourier transform is

\begin{equation}\label{eqn:qmTwoL8:130}
E(\omega) = \frac{e^{-\frac{1}{4} T^2 (\omega_0+\omega )^2}}{2 \sqrt{\frac{2}{T^2}}}+\frac{e^{\omega_0 T^2 \omega -\frac{1}{4} T^2 (\omega_0+\omega )^2}}{2 \sqrt{\frac{2}{T^2}}}
\end{equation}

This is illustrated, again for \(\omega_0 = 10\), and \(T=1\), in \cref{fig:qmTwoL8:FTgaussianWavePacket}

\imageFigure{../../figures/phy456/FTgaussianWavePacket}{FTgaussianWavePacket}{fig:qmTwoL8:FTgaussianWavePacket}{0.2}

where we see the expected Gaussian result, since the Fourier transform of a Gaussian is a Gaussian.

FIXME: not sure what the point of this was?

\section{Sudden perturbations}
\index{sudden perturbation}

Given our wave equation

\begin{equation}\label{eqn:qmTwoL8:150}
i \Hbar \ddt{} \ket{\psi(t)} = H(t) \ket{\psi(t)}
\end{equation}

and a sudden perturbation in the Hamiltonian, as illustrated in \cref{fig:qmTwoL8:suddenStepHamiltonian}

\imageFigure{../../figures/phy456/suddenStepHamiltonian}{Sudden step Hamiltonian}{fig:qmTwoL8:suddenStepHamiltonian}{0.2}

Consider \(H_0\) and \(H_F\) fixed, and decrease \(\Delta t \rightarrow 0\).  We can formally integrate \eqnref{eqn:qmTwoL8:150}

\begin{equation}\label{eqn:qmTwoL8:150b}
\ddt{} \ket{\psi(t)} = \inv{i \Hbar} H(t) \ket{\psi(t)}
\end{equation}

For
\begin{equation}\label{eqn:qmTwoL8:150e}
\ket{\psi(t)} -\ket{\psi(t_0)}
 = \inv{i \Hbar} \int_{t_0}^t H(t') \ket{\psi(t')} dt'.
\end{equation}

While this is an exact solution, it is also not terribly useful since we do not know \(\ket{\psi(t)}\).  However, we can select the small interval \(\Delta t\), and write

\begin{equation}\label{eqn:qmTwoL8:150c}
\ket{\psi(\Delta t/2)} =
\ket{\psi(-\Delta t/2)}
+ \inv{i \Hbar} \int_{t_0}^t H(t') \ket{\psi(t')} dt'.
\end{equation}

Note that we could use the integral kernel iteration technique here and substitute \(\ket{\psi(t')} = \ket{\psi(-\Delta t/2)}\) and then develop this, to generate a power series with \((\Delta t/2)^k\) dependence.  However, we note that \eqnref{eqn:qmTwoL8:150c} is still an exact relation, and if \(\Delta t \rightarrow 0\), with the integration limits narrowing (provided \(H(t')\) is well behaved) we are left with just

\begin{equation}\label{eqn:qmTwoL8:180}
\ket{\psi(\Delta t/2)} = \ket{\psi(-\Delta t/2)}
\end{equation}

Or
\begin{equation}\label{eqn:qmTwoL8:200}
\ket{\psi_{\text{after}}} = \ket{\psi_{\text{before}}},
\end{equation}

provided that we change the Hamiltonian fast enough.  On the surface there appears to be no consequences, but there are some very serious ones!

\makeexample{Harmonic oscillator}{l8:ex1}{

Consider our harmonic oscillator Hamiltonian, with

\begin{equation}\label{eqn:qmTwoL8:220}
\begin{aligned}
H_0 &= \frac{P^2}{2m} + \inv{2} m \omega_0^2 X^2 \\
H_F &= \frac{P^2}{2m} + \inv{2} m \omega_F^2 X^2
\end{aligned}
\end{equation}

Here \(\omega_0 \rightarrow \omega_F\) continuously, but \textunderline{very quickly}.  In effect, we have tightened the spring constant.  Note that there are cases in linear optics when you can actually do exactly that.

Imagine that \(\ket{\psi_{\text{before}}}\) is in the ground state of the harmonic oscillator as in \cref{fig:qmTwoL8:suddenHamiltonianPertubationHO}

\imageFigure{../../figures/phy456/suddenHamiltonianPertubationHO}{Harmonic oscillator sudden Hamiltonian perturbation}{fig:qmTwoL8:suddenHamiltonianPertubationHO}{0.2}

and we suddenly change the Hamiltonian with potential \(V_0 \rightarrow V_F\) (weakening the ``spring'').  Professor Sipe gives us a graphical demo of this, by impersonating a constrained wavefunction with his arms, doing weak chicken-flapping of them.  Now with the potential weakened, he wiggles and flaps his arms with more freedom and somewhat chaotically.  His ``wave function'' arms are now bouncing around in the new limiting potential (initially over doing it and then bouncing back).

We had in this case the exact relation

\begin{equation}\label{eqn:qmTwoL8:240}
H_0 \ket{\psi_0^{(0)}} = \inv{2} \Hbar \omega_0 \ket{\psi_0^{(0)}}
\end{equation}

but we also have
\begin{equation}\label{eqn:qmTwoL8:260}
\ket{\psi_{\text{after}}} = \ket{\psi_{\text{before}}} = \ket{\psi_0^{(0)}}
\end{equation}

and
\begin{equation}\label{eqn:qmTwoL8:280}
H_F \ket{\psi_n^{(f)}} = \inv{2} \Hbar \omega_F \left( n + \inv{2} \right) \ket{\psi_n^{(f)}}
\end{equation}

So
\begin{equation}\label{eqn:qmTwoL8:510}
\begin{aligned}
\ket{\psi_{\text{after}}}
&=
\ket{\psi_0^{(0)}} \\
&=
\sum_n \ket{\psi_n^{(f)}}
\mathLabelBox{\braket{\psi_n^{(f)}}{\psi_0^{(0)}} }{\(c_n\)} \\
&=
\sum_n c_n \ket{\psi_n^{(f)}}
\end{aligned}
\end{equation}

and at later times

\begin{equation}\label{eqn:qmTwoL8:530}
\begin{aligned}
\ket{\psi(t)^{(f)}}
&=
\ket{\psi_0^{(0)}} \\
&=
\sum_n c_n e^{i \omega_n^{(f)} t} \ket{\psi_n^{(f)}},
\end{aligned}
\end{equation}

whereas

\begin{equation}\label{eqn:qmTwoL8:550}
\begin{aligned}
\ket{\psi(t)^{(o)}}
&=
e^{i \omega_0^{(0)} t} \ket{\psi_0^{(0)}},
\end{aligned}
\end{equation}

So, while the wave functions may be exactly the same after such a sudden change in Hamiltonian, the dynamics of the situation change for all future times, since we now have a wavefunction that has a different set of components in the basis for the new Hamiltonian.  In particular, the evolution of the wave function is now significantly more complex.

FIXME: plot an example of this.
}

\shipoutAnswer

\section{Adiabatic perturbations}
\index{adiabatic perturbation}

This is treated in \S 17.5.2 of the text \citep{desai2009quantum}.

I wondered what Adiabatic meant in this context.  The usage in class sounds like it was just ``really slow and gradual'', yet this has a definition ``\href{http://www.thefreedictionary.com/adiabatic}{Of, relating to, or being a reversible thermodynamic process that occurs without gain or loss of heat and without a change in entropy}''.  Wikipedia \citep{wiki:AdiabaticTheorem} appears to confirm that the QM meaning of this term is just ``slow'' changing.

This is the reverse case, and we now vary the Hamiltonian \(H(t)\) \textunderline{very slowly}.

\begin{equation}\label{eqn:qmTwoL8:150f}
\ddt{} \ket{\psi(t)} = \inv{i \Hbar} H(t) \ket{\psi(t)}
\end{equation}

We first consider only non-degenerate states, and at \(t = 0\) write

\begin{equation}\label{eqn:qmTwoL8:300}
H(0) = H_0,
\end{equation}

and
\begin{equation}\label{eqn:qmTwoL8:320}
H_0 \ket{\psi_s^{(0)}} = E_s^{(0)} \ket{\psi_s^{(0)}}
\end{equation}

Imagine that at each time \(t\) we can find the ``instantaneous'' energy eigenstates

\begin{equation}\label{eqn:qmTwoL8:340}
H(t) \ket{\hat{\psi}_s(t)} = E_s(t) \ket{\hat{\psi}_s(t)}
\end{equation}

These states do not satisfy Schr\"{o}dinger's equation, but are simply solutions to the eigen problem.  Our standard strategy in perturbation is based on analysis of

\begin{equation}\label{eqn:qmTwoL8:360}
\ket{\psi(t)} = \sum_n c_n(t) e^{- i \omega_n^{(0)} t} \ket{\psi_n^{(0)} },
\end{equation}

Here instead

\begin{equation}\label{eqn:qmTwoL8:380}
\ket{\psi(t)} =
\sum_n b_n(t) \ket{\hat{\psi}_n(t)}
,
\end{equation}

we will expand, not using our initial basis, but instead using the instantaneous kets.  Plugging into Schr\"{o}dinger's equation we have

\begin{equation}\label{eqn:qmTwoL8:570}
\begin{aligned}
H(t) \ket{\psi(t)}
&= H(t) \sum_n b_n(t) \ket{\hat{\psi}_n(t)} \\
&= \sum_n b_n(t) E_n(t) \ket{\hat{\psi}_n(t)}
\end{aligned}
\end{equation}

This was complicated before with matrix elements all over the place.  Now it is easy, however, the time derivative becomes harder.  Doing that we find

\begin{equation}\label{eqn:qmTwoL8:590}
\begin{aligned}
i \Hbar \ddt{} \ket{\psi(t)}
&=
i \Hbar
\ddt{} \sum_n b_n(t) \ket{\hat{\psi}_n(t)} \\
&=
i \Hbar
\sum_n
\ddt{b_n(t)} \ket{\hat{\psi}_n(t)}
+ \sum_n b_n(t) \ddt{} \ket{\hat{\psi}_n(t)} \\
&= \sum_n b_n(t) E_n(t) \ket{\hat{\psi}_n(t)}
\end{aligned}
\end{equation}

We bra \(\bra{\hat{\psi}_m(t)}\) into this

\begin{equation}\label{eqn:qmTwoL8:400}
i \Hbar
\sum_n
\ddt{b_n(t)}
\braket{\hat{\psi}_m(t)}{\hat{\psi}_n(t)}
+ \sum_n b_n(t)
\bra{\hat{\psi}_m(t)}
\ddt{} \ket{\hat{\psi}_n(t)}
= \sum_n b_n(t) E_n(t) \braket{\hat{\psi}_m(t)}{\hat{\psi}_n(t)} ,
\end{equation}

and find

\begin{equation}\label{eqn:qmTwoL8:410}
i \Hbar
\ddt{b_m(t)}
+
\sum_n b_n(t)
\bra{\hat{\psi}_m(t)}
\ddt{} \ket{\hat{\psi}_n(t)}
= b_m(t) E_m(t)
\end{equation}

If the Hamiltonian is changed \textunderline{very very} slowly in time, we can imagine that \(\ket{\hat{\psi}_n(t)}'\) is also changing very very slowly, but we are not quite there yet.  Let us first split our sum of bra and ket products

\begin{equation}\label{eqn:qmTwoL8:430}
\sum_n b_n(t)
\bra{\hat{\psi}_m(t)}
\ddt{} \ket{\hat{\psi}_n(t)}
\end{equation}

into \(n \ne m\) and \(n = m\) terms.  Looking at just the \(n = m\) term

\begin{equation}\label{eqn:qmTwoL8:450}
\bra{\hat{\psi}_m(t)}
\ddt{} \ket{\hat{\psi}_m(t)}
\end{equation}

we note

\begin{equation}\label{eqn:qmTwoL8:610}
\begin{aligned}
0
&=
\ddt{} \braket{\hat{\psi}_m(t)}{\hat{\psi}_m(t)} \\
&=
\left( \ddt{} \bra{\hat{\psi}_m(t)} \right) \ket{\hat{\psi}_m(t)}
+ \bra{\hat{\psi}_m(t)} \ddt{} \ket{\hat{\psi}_m(t)} \\
\end{aligned}
\end{equation}

Something plus its complex conjugate equals 0

\begin{equation}\label{eqn:qmTwoL8:470}
a + i b + (a + i b)^\conj = 2 a = 0 \implies a = 0,
\end{equation}

so \(\bra{\hat{\psi}_m(t)} \ddt{} \ket{\hat{\psi}_m(t)}\) must be purely imaginary.  We write

\begin{equation}\label{eqn:qmTwoL8:490}
\bra{\hat{\psi}_m(t)} \ddt{} \ket{\hat{\psi}_m(t)} = -i \Gamma_s(t),
\end{equation}

where \(\Gamma_s\) is real.



