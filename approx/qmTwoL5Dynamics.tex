%
% Copyright � 2013 Peeter Joot.  All Rights Reserved.
% Licenced as described in the file LICENSE under the root directory of this GIT repository.
%
\section{Review of dynamics}
\index{dynamics}

We want to move on to time dependent problems.  In general for a time dependent problem, the answer follows provided one has solved for \textunderline{all} the perturbed energy eigenvalues.  This can be laborious (or not feasible due to infinite sums).

Before doing this, let us review our dynamics as covered in \S 3 of the text \citep{desai2009quantum}.

\paragraph{Schr\"{o}dinger and Heisenberg pictures}

Our operator equation in the Schr\"{o}dinger picture is the familiar

\begin{equation}\label{eqn:qmTwoL5:220}
i \Hbar \frac{d}{dt} \ket{\psi_s(t)} = H \ket{\psi_s(t)}
\end{equation}

and most of our operators \(X, P, \cdots\) are time independent.

\begin{equation}\label{eqn:qmTwoL5:240}
\expectation{O}(t) =
\bra{\psi_s(t)} O_s
\ket{\psi_s(t)}
\end{equation}

where \(O_s\) is the operator in the Schr\"{o}dinger picture, and is non time dependent.

Formally, the time evolution of any state is given by

\begin{equation}\label{eqn:qmTwoL5:260}
\ket{\psi_s(t)}
e^{-i H t/\Hbar}
\ket{\psi_s(0)} = U(t, 0) \ket{\psi_s(0)}
\end{equation}

so the expectation of an operator can be written
\begin{equation}\label{eqn:qmTwoL5:280}
\expectation{O}(t) =
\bra{\psi_s(0)}
e^{i H t/\Hbar}
O_s
e^{-i H t/\Hbar}
\ket{\psi_s(0)}.
\end{equation}

With the introduction of the Heisenberg ket

\begin{equation}\label{eqn:qmTwoL5:300}
\ket{\psi_H} = \ket{\psi_s(0)},
\end{equation}

and Heisenberg operators

\begin{equation}\label{eqn:qmTwoL5:320}
O_H = e^{i H t/\Hbar} O_s e^{-i H t/\Hbar},
\end{equation}

the expectation evolution takes the form

\begin{equation}\label{eqn:qmTwoL5:340}
\expectation{O}(t) =
\bra{\psi_H}
O_H
\ket{\psi_H}.
\end{equation}

Note that because the Hamiltonian commutes with its exponential (it commutes with itself and any power series of itself), the Hamiltonian in the Heisenberg picture is the same as in the Schr\"{o}dinger picture

\begin{equation}\label{eqn:qmTwoL5:360}
H_H = e^{i H t/\Hbar} H e^{-i H t/\Hbar} = H.
\end{equation}

\paragraph{Time evolution and the Commutator}

%The equivalent of the Hamiltonian equations for the time evolution of a particle in classical physics
Taking the derivative of \eqnref{eqn:qmTwoL5:320} provides us with the time evolution of any operator in the Heisenberg picture

\begin{equation}\label{eqn:qmTwoL5Dynamics:400}
\begin{aligned}
i \Hbar \frac{d}{dt} O_H(t)
&=
i \Hbar \frac{d}{dt} \left(
e^{i H t/\Hbar} O_s e^{-i H t/\Hbar}
\right) \\
&=
i \Hbar \left(
\frac{i H}{\Hbar} e^{i H t/\Hbar} O_s e^{-i H t/\Hbar}
+
e^{i H t/\Hbar} O_s e^{-i H t/\Hbar} \frac{-i H}{\Hbar}
\right) \\
&=
\left(
-H O_H
+
O_H H
\right).
\end{aligned}
\end{equation}

We can write this as a commutator

\begin{equation}\label{eqn:qmTwoL5:380}
i \Hbar \frac{d}{dt} O_H(t) = \antisymmetric{O_H}{H}.
\end{equation}

\paragraph{Summarizing the two pictures}

\begin{equation}\label{eqn:qmTwoL5Dynamics:420}
\begin{aligned}
\text{Schr\"{o}dinger picture} &\qquad \text{Heisenberg picture} \\
i \Hbar \frac{d}{dt} \ket{\psi_s(t)} = H \ket{\psi_s(t)} &\qquad i \Hbar \frac{d}{dt} O_H(t) = \antisymmetric{O_H}{H} \\
\bra{\psi_s(t)} O_S \ket{\psi_s(t)} &= \bra{\psi_H} O_H \ket{\psi_H} \\
\ket{\psi_s(0)} &= \ket{\psi_H} \\
O_S &= O_H(0)
\end{aligned}
\end{equation}


